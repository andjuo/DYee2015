\documentclass[a4paper]{article}

\pdfoutput=1
%\usepackage{jheppub} 

\usepackage{lineno}
\usepackage[pdftex]{graphicx}
%\graphicspath{{dir-figs/}}

\usepackage{stmaryrd,xcolor}
\newcommand{\textred}[1]{{\color{red!70!black}#1}}
\newcommand{\textgreen}[1]{{\color{green!50!black}#1}}
\newcommand{\textblue}[1]{{\color{blue!90!black}#1}}

\title{Calculation of the covariance matrix for the Drell-Yan measurement}
\author{A.~Juodagalvis\\[2mm]
\small
Vilnius University, Institute of Theoretical Physics and Astronomy, \\
\small A.\ Go\v{s}tauto st.\ 12, Vilnius 01108, Lithuania}

\begin{document}
\linenumbers

%\abstract{Information about the needed data structures.}
\maketitle

The aim of this summary is to provide information about the needed
data structures. All quantities that involve event count should be
provided as a 1D histogram containing both the event count $N$ and the
associated uncertainty. In the case of experimental data, the
uncertainty is simply the Poissonian uncertainty ($\sqrt{N}$). The
simulated event count may have a different uncertainty, depending on
how the event count was obtained. For example, the background
estimation uncertainty may be different from the Poissonian prediction.

The 1D Drell-Yan cross section measurement provides the cross section
values in bins of mass

\begin{equation}
\frac{d\sigma}{dM}\ =\ {\cal F}^{-1}_\mathrm{FSR}
\left\{
\frac{
  {\cal F}^{-1}_\mathrm{resol}\{ N_\mathrm{obs.yield} - N_\mathrm{bkg.yield} \}
}{
  A \varepsilon \rho
}
\right\},
\end{equation}
where ${\cal F}^{-1}_\mathrm{X}$ denotes the unfolding correction for
the detector ($X=\mathrm{resol}$) and FSR ($X=\mathrm{FSR}$) effects,
$N_\mathrm{obs.yield}$ is the measured event yield,
$N_\mathrm{bkg.yield}$ is the estimated background yield, $A$ is the
event acceptance, $\varepsilon$ is the event efficiency, and $\rho$ is
the efficiency correction factor.

\medskip

{\bf Input 1}. Measured event yield distribution $N_\mathrm{obs.yield}$ as
a 1D histogram.

\medskip

{\bf Input 2}. Background event yield distributions subdivided by
sources (1D distributions)

\begin{equation}
N_\mathrm{bkg}\ =\ \sum\limits_\mathrm{source} N_\mathrm{bkg.yield,source}
\end{equation}

\medskip

{\bf Input 3}. Since the correction-related uncertainties are
evaluated by varying certain input parameters (e.g.\ individual scale
factor values), a compressed dataset of simulated \textit{signal}
events is needed. Try to keep only relevant events.
It would probably be something like

\begin{equation}
\begin{tabular}{cl}\\
\phantom{xxx} &
$
  \{
  \mathrm{four\_momentum\_of\_lepton}_{1,\mathrm{reco}}$,\\
&  $\mathrm{four\_momentum\_of\_lepton}_{2,\mathrm{reco}}$,\\
&  $\mathrm{selected\_flag},$\\
&  $\mathrm{four\_momentum\_of\_lepton}_{1,\mathrm{gen,postFSR}},$\\
&  $\mathrm{four\_momentum\_of\_lepton}_{2,\mathrm{gen,postFSR}},$\\
&  $\mathrm{four\_momentum\_of\_lepton}_{1,\mathrm{gen,preFSR}},$\\
&  $\mathrm{four\_momentum\_of\_lepton}_{2,\mathrm{gen,preFSR}},$\\
&  $\mathrm{event\_weight}
  \},$
\end{tabular}
\end{equation}

\noindent
where ``selected\_flag'' marks whether the event passes the event
selection requirements for $p_T$, $\eta$, isolation etc.

This dataset should allow picking the correct lepton momentum/energy
or efficiency scale correction for each event, and calculate any
simulation-based correction (like $\varepsilon$ and $A$).

The acceptable cut-off for the dataset is
$M_{\ell\ell,\mathrm{gen,preFSR}}\leq15$~GeV, if
$M_{\ell\ell,\mathrm{reco}}<15$~GeV. No selection cuts should be
applied (only flaged for passing), so that this dataset could be used
for all corrections.

\medskip

{\bf Input 4}. Tables for lepton momentum/energy scale correction (in
a suitable format).

\medskip

{\bf Input 5}. Tables for the efficiency scale correction from
tag-and-probe. Include each type of $\rho$ factor with the associated
uncertainty.

\medskip

{\bf Input 6}. Distributions for cross checks. For example,
RooBayesResponse objects for the detector resolution and FSR
unfoldings. Event efficiency scale $\rho$, efficiency $\varepsilon$,
acceptance $A$ distributions. Yield distribution without lepton
momentum/energy scale correction.
Final distribution of the cross section.

\end{document}

