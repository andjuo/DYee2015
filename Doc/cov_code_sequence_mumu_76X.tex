\documentclass[a4paper]{article}

\pdfoutput=1
%\usepackage{jheppub} 

\usepackage{lineno}
\usepackage[pdftex]{graphicx}
\usepackage{xstring} % StrSubstitute
\usepackage{ulem} % sout
%\graphicspath{{dir-figs/}}

\usepackage{stmaryrd,xcolor}
\newcommand{\textred}[1]{{\color{red!70!black}#1}}
\newcommand{\textgreen}[1]{{\color{green!50!black}#1}}
\newcommand{\textblue}[1]{{\color{blue!90!black}#1}}
\newcommand{\lumi}[1]{{\cal L}_{\mathrm{#1}}}
\newcommand{\avgrho}{\langle\rho\rangle}
\newcommand{\tab}{$\phantom{.\hspace{1cm}.}$}

\def\myund#1{%
  \saveexpandmode\expandarg
  \IfSubStr{#1}{_}{%
   \StrSubstitute{#1}{_}{\_}}{#1}%
  \restoreexpandmode
}


\title{Calculation of the covariance matrix for the Drell-Yan measurement in the muon channel: Code sequence}
\author{A.~Juodagalvis\\[2mm]
\small
Vilnius University, Institute of Theoretical Physics and Astronomy, \\
\small Saul{\. e}tekio al.\ 3, Vilnius 10222, Lithuania}

\begin{document}
\linenumbers

%\abstract{Information about the needed data structures.}
\maketitle

\section{Dimuon covariance calculation}

Explanation of the code sequence used to derive the covariance
matrices of the 1D measurement in the muon channel. Explanation is
based on the provided data with the tag ``76X''.

The 1D Drell-Yan cross section measurement provides the cross section
values in bins of mass

\begin{equation}
\frac{d\sigma}{dM}\ =\ {\cal F}^{-1}_\mathrm{FSR}
\left\{
\frac{
  {\cal F}^{-1}_\mathrm{resol}\{ N_\mathrm{obs.yield} - N_\mathrm{bkg.yield} \}
}{
  A \varepsilon \rho \lumi{int}
}
\right\},
\end{equation}
where ${\cal F}^{-1}_\mathrm{X}$ denotes the unfolding correction for
the detector ($X=\mathrm{resol}$) and FSR ($X=\mathrm{FSR}$) effects,
$N_\mathrm{obs.yield}$ is the measured event yield,
$N_\mathrm{bkg.yield}$ is the estimated background yield, $A$ is the
event acceptance, $\varepsilon$ is the event efficiency, $\rho$ is
the efficiency correction factor, and $\lumi{int}$ is the total
integrated luminosity.

Due to the differences in the HLT menus v.4.2 and v.4.3, the
measurement in the muon channel is subdivided into two run periods.
The background yield is estimated from all data, thus the cross
section measurement is calculated as
\begin{equation}
\frac{d\sigma}{dM} =
\frac{1}{\lumi{int.tot}}
{\cal F}_{\mathrm{FSR}^{-1}}
\left\{
  \sum\limits_{\mathrm{HLTvX: X=4.2,4.3}}
  \frac{%
    {\cal F}_{\mathrm{resol}}^{-1}
    \left\{
    N_\mathrm{obs.yield,HLTvX} -
    \frac{\lumi{int,HLTvX}}{\lumi{int,tot}} N_\mathrm{bkg.yield}
    \right\}
  }
  {A\varepsilon \rho_{HLTvX}}
\right\},
\end{equation}
where $\lumi{int,HLTv4.2}=848.104$~pb${}^{-1}$ and
$\lumi{int,HLTv4.3}=1917.125$~pb${}^{-1}$.

\medskip

All relevant macros are placed in subdirectory {\sf Covariance}. The
header file \myund{crossSection.h}\ contains main class definitions
used when deriving the covariance matrix: \myund{CrossSection_t}\ and
\myund{MuonCrossSection_t}. The header file \myund{inputs.h}\
contains many auxiliary functions. The header file
\myund{DYmm13TeV.h}\ defines class \myund{DYmm13TeV_t}\ that has a
description of the Drell-Yan event file in the muon channel containing
data in the ROOT format.

\textbf{First, we need to convert the input files.} The macro is called
\myund{createCSInput_DYmm13TeV.C}. One needs to specify the source
path variable \myund{srcPath}\ to point to relevant location.
The variables (inpVer and inpVerTag, defined in
crossSection.h) have to be updated as well. The macro should be run
with pre-compilation:
\begin{verbatim}
.x createCSInput_DYmm13TeV.C+(doSave)
\end{verbatim}
The variable doSave is 1 for saving the resulting ROOT file. The
default value is 0 (no save).

The first input file is
\myund{Input1/ROOTFile_Histograms_Data.root}. The histograms are
\myund{h_data_HLTv4pX}\ and \myund{h_yield_HLTv4pX}, where X=2 and
3. They are loaded under names \myund{h1Yield_4X}\ and
\myund{h1Signal_4X}. These histograms correspond to
$N_\mathrm{obs.yield,HLTvX}$ and $(N_\mathrm{obs.yield,HLTvX} -
      {\lumi{int,HLTvX}}/{\lumi{int,tot}} N_\mathrm{bkg.yield})$.
% 
Note that no events were registered in the mass range 1500-3000~GeV
during the run period HLTv4.2.

The second input file
\myund{Input2/ROOTFile_Histograms_Bkg.root}\ contains several
backgrounds. The expected ones are given in histograms \myund{h_XX},
where 'XX' stands for ZZ, WZ, WW, ttbar (TTbar), DYtautau, tW (TW),
WJets, QCD. Those backgrounds are listed as the enumerated type
\myund{TBkg_t}\ in \myund{crossSection.h}. Their variable names are
\myund{_bkgXX}, where 'XX' stands for the background name (note, that
ttbar and tW are named \myund{_bkgTTbar}\ and \myund{_bkgTW}).

If the variable \myund{checkBackgrounds}\ is set to 1, the code will
try to verify whether the backgrounds are reasonable. The code takes
the yield histogram and subtracts the signal histogram. This should
yield the background distribution (after corrections for the run
luminosities).
Inputs v1 had
improperly normalized ZZ, WZ, and WW backgrounds. This trouble is
fixed in version 76X. (The following problem was fixed on June 3, 2016:
\sout{However, here is a non-trivial mismatch between
$(h_\mathrm{yield}-h_\mathrm{signal})/sum_{\mathrm{bkg}}(h_\mathrm{bkg})$. It
varied between $-3\%$ in the first mass bin up to $+5\%$ in the mass
bin 42, 1000-1500~GeV. The difference is the same in both versions
HLTv4.X.})

The third input file is
\myund{Input6/ROOTFile_Input6_CrossCheck.root}. It contains several
objects of type \myund{TGraphAsymmErrors}: acceptance $A$
(\myund{g_Acc}), efficiency $\varepsilon$ (\myund{g_Eff}),
$A\times\varepsilon$ (\myund{g_AccEff}), efficiency correction scale
factors $\rho$ (\myund{g_EffSF_HLTv4pX}, where X=2 and 3). Two objects
of type \myund{RooUnfoldResponse}\ give the corrections of the
detector resolution and the final-state radiation
(\myund{UnfoldRes_DetectorResol}\ and \myund{UnfoldRes_FSR},
respectively). The unfolded data is stored in the histogram
\myund{h_DiffXsec_Data}. All graphs are converted to histograms with a
symmetrized uncertainty.

The loaded values are assigned to an instance of
\myund{MuonCrossSection_t}\ and saved to a file
\myund{``cs_DYmm_13TeV''+inpVerTag}. Actually, three files will be
produced: \myund{cs_DYmm_13TeV_76X_csA.root},
\myund{cs_DYmm_13TeV_76X_csB.root}, and\linebreak
\myund{cs_DYmm_13TeV_76X_cs.root}. (Here the tag \myund{'_76X'}\ was
used.)

Note from June 3, 2016. The current version of 76X set has fixed the
background mismatch. However, the difference between the calculated
cross section distribution and the distribution provided by Kyeongpil
amounts to $2.439\%$. Since this difference is almost flat, it may be
due to some unaccounted factor that has changed. The variation takes
place only in the mass bins 830-1000~GeV (the relative difference is
$2.567\%$ and 1000-1500~GeV ($1.135\%$).

I am not sure when the change occured, but later I found that the
backgrounds from ZZ and WZ should by multiplied by the corresponding
event efficiency scale factor in each mass bin.

Note from May 5, 2017. Kyeongpil provided a new set of inputs. The
histograms with the ZZ background had names \myund{h_ZZ_HLTv4p2}\ and
\myund{h_ZZ_HLTv4p3}, and the WZ background had the corresponding
names with WZ. I started to snoop around and found this need (encoded
in \myund{MuonCrossSection_t::sampleRndVec}) to rescale the
backgrounds by $\lumi{HLT period}\times\rho_{HLT period}$. Other
backgrounds are normalized to $\lumi{tot}$.

In fact, Kyeongpil explained that the scale factor is not the average
one. It was applied per-event, which explains the residual difference
of up to $1.5\%$, after I use the named factors. The macro name is
\myund{checkDYmmInputs_Bkg.C}. If I follow the procedure that
Kyeongpil delineated, the sum of the properly-weighted backgrounds
equals the difference between measured and signal yields:
\begin{eqnarray}
N_\mathrm{sig,HLTv}-N_\mathrm{obs,HLTv} &=& N_{ZZ,\textrm{HLTv}}
  + N_{WZ,\textrm{HLTv}} \nonumber \\
 &+&
\sum\limits_{bkg\in\{WW,t{\bar t},DY\tau\tau,tW,Wjets,QCD\}}
  \frac{\lumi{HLTv}}{\lumi{tot}} N_{bkg}.
\end{eqnarray}
The use of the averaged $ZZ$ and $WZ$ background gives deviation of
$0.34-0.35\%$ in the mass bins 86-91-96~GeV. The final cross section
agrees well (the difference is negligible), with an exception for the
mass bin 1000-1500~GeV, where the relative difference is $0.8\%$, if
the flag ``removeNegativeSignal'' is not set to 1. If negative signal
is removed, the difference is negligible in all mass bins.

\textbf{Secondly, the randomized event-effciency scale factors should
  be prepared.}  The code \myund{process_DYmm_RECO.C}\ reads the
``compressed'' n-tuple and produces detector resolution and the
efficiency scale factor corrections. It also saves 1D distributions of
the pre-unfolded and unfolded yields, and copies the single-muon
efficiencies from the provided input5. It also produces distributions
needed to calculate the event-efficiency scale factor in an optimized
way. The output is saved to a file
\myund{dymm_test_RECO_$<$InpVersion$>$.root}.

The randomized ensemble of the event-efficiency scale factors is
produced by the code \myund{calcRhoRndVec.C}. It reads the file
\myund{dymm_test_RECO_$<$InpVersion$>$.root}. It also compares the
scale factors to the provided values. The random ensemble is saved to
a file
\myund{dir-Rho-$<$InpVersion$>$/dymm_rhoRndVec_$<$InpVersion$>$_$<$NSamples$>$.root}.

\textbf{Thirdly, the covariances are being derived.} The macro is
called \myund{studyDYmmCS.C}. This macro has several input parameters:
the varied variable \myund{TVaried_t}\ (the enum type is defined in
crossSection.h); the number of toy distributions nSample; and a flag
whether the resulting ROOT file should be saved, doSave. Before first
use, the macro has to be updated to indicate the correct version of
the input files by specifying the right values of the variables inpVer
and inpVerTag. Those are equivalent to the ones defined in
\myund{createCSInput_DYmm13TeV.C}.

Currently the macro \myund{studyDYmmCS.C}\ can vary \myund{_varYield},
\myund{_varBkg}, \myund{_varBkgXS}, \myund{_varDetRes}\ (statistical
uncertainty of the detector resolution response matrix),
\myund{_varFSRRes}\ (statistical uncertainty of the FSR response
matrix), \myund{_varEff}, \myund{_varAcc}, and \myund{_varEffAcc}.
The unimplemented variables are: \myund{_varSig}, \myund{_varRho}.
(Note: \myund{_varRhoFile}\ is implemented to use the ensemble
produced by the code \myund{calcRhoRndVec.C}.)

Note June 8, 2016: It seems that the uncertainties from \myund{_varEff} and
\myund{_varAcc} are smaller than the ones provided for
\myund{_varEffAcc}. That is, the combined uncertainty of
$\varepsilon\times A$ is smaller than provided by KL.

The code \myund{calcRhoRndVec.C}\ compares the calculated event
efficiency scale factor to the provided event efficiency scale factor.

Note May 9, 2017: It was not documented, but for yield uncertainty
\myund{_varYieldPoisson}\ should be used.

\subsection{Verification}

The code \myund{createCSInput_DYmm13TeV.C}\ verifies several things.
1)~The provided backgrounds by comparing the sum of
the provided background counts to the difference of the provided
observed yield and the signal yield (observed yield with the
background subtracted). 2)~The uncertainties of the combined quantities
$\epsilon_\mathrm{MC}\times A$. It compares whether those
uncertainties are equal or smaller than the uncertainties of the
combination. Since
$\epsilon_\mathrm{MC}=N_\mathrm{sel,inAcc,postFSR}/N_\mathrm{inAcc,postFSR}$
and $A=N_\mathrm{inAcc,postFSR}/N_\mathrm{postFSR}$, the combination
$\epsilon_\mathrm{MC}\times
A=N_\mathrm{sel,inAcc,postFSR}/N_\mathrm{postFSR}$ can have a smaller
uncertainty. This, however, is not exploited.
3)~The provided cross section and the inputs by calculating the final
cross section from the provided inputs and comparing it to the
provided final cross section.

The event-efficiency scale factors are tested by two codes.  The code
\myund{calcRhoRndVec.C}\ tests the output of the code
\myund{processDYmm_RECO.C}\ by comparing the produced scale factor to
the provided values.  The code \myund{processDYmm_avgRho2.C}\ verifies
that the event-efficiency scale factors derived by using the event
space correspond to the provided scale factor distributions.

The code \myund{processDYmm.C}\ reads the ``compressed'' n-tuple and
produces several distributions that lead to the efficiency, acceptance
and the FSR corrections.
 The output is saved to a file
\myund{dymm_test_$<$InpVersion$>$.root}.
This file is later used by a complicated code
\myund{study_FSRResp.C}\ that compares several items.

The code \myund{study_FSRResp.C}\ reads the main result file
\myund{cs_DYmm_13TeV_$<$ShortInpVersion$>$_cs.root}\ (fnameMain) to
get post-FSR and pre-FSR distributions h1PostFSR and h1PreFSR
(variables nameh1preUnfData and nameh1unfData). Those distributions
are obtained from the provided measured data by applying the provided
corrections.  The FSR-correction may be read from several sources
(including the main result file fnameMain).  The file and the object
names are given by variables fnameResp and nameResp.  For 76X input
version, the object \myund{rooUnf_fsrResp} is taken from a file
\myund{dymm_test_$<$InpVersion$>$.root}\ that is produced by the code
\myund{processDYmm.C}. The simulated distributions are read-in from a
file defined by fnameMC and given by the variables nameh1measMC and
nameh1trueMC (\myund{h1_postFSR_Mweighted}\ and
\myund{h1_preFsr_Mweighted}). To be able to test the early result in
the electron channel by using the migration matrix, a variable was
introduced nameh2migrationMatrix. The code can also read-in the
covariance from the randomized response. The variable is covFName and
nameh2cov with the values \myund{cov_mumu_varFSRRes_10000.root}\ and
h2Cov.

% ----------------------------------
\section{Dielectron covariance calculation}
\label{sect-DielectronCov}

The code sequence for the dielectrons:
(1)~\myund{processDYee_dressed.C},
(2)~\myund{createCSInput_DYee13TeV.C},
(3)~\myund{calcRhoRndVec_ee.C},
(4)~\myund{studyDYeeCS.C} (note: the script
\myund{run_studyDYmmCS.sh}\ was updated to call also studyDYeeCS.C macro).

(Added Jan 19, 2017)\\\noindent The code
\myund{calcRhoRndVec_ee_syst.C}\ is used to produce several kinds of
event efficiency systematic uncertainties by use of ``alternative
tables'' of the efficiencies derived changing the selection or the fit
(e.g.\ sigPdf, bkgPdf, tagDef - for data, NLOvsLO - for MC). The use
is \myund{calcRhoRndVec_ee_syst.C+(1,0,1,1111)}.  This produces the
files needed for systematic evaluations by code
\myund{studyDYeeCS.C+(_varRhoSyst,0,0)}.

The study of event efficiency scale factor systematic uncertainty
covariance is continued in Section \ref{sect-effSFsyst} from July 27,
2017.

% ----------------------------------
\section{Calculation of the statistical uncertainty covariance matrix}

Because of the fluctuations in the correlation matrix from the
measured yield statistical uncertainty, a sequence of codes was
developed to study the stability.

For the muon channel:\\
Step 1: \myund{studyDYmmCS.C+(_varYield,Nsample,2)}\\
Step 2: \myund{studyDYmmCScovYield.C+(Nsample,iMethod,1,plotPeriodA)}\\
Step 3: \myund{studyDYmmCScovYieldMethods.C+(Nsample)}

For the electron channel:\\
Step 1: \myund{studyDYeeCS.C+(_varYield,Nsample,2)}\\
Step 2:
\myund{studyDYeeCScovYield.C+(Nsample,iMethod,1,plotMassDistr)}\\
Step 3: \myund{studyDYeeCScovYieldMethods.C+(Nsample)}

The value of Nsample determines how large the Gauss-randomized
ensemble should be created. The number 2 in the code studyDYxxCS.C
argument list indicates that saving of a slimmed version is requested
(the ROOT file will contain a tag 'slim'). If changed to 0, saving is
not done. The value of 1 saves ``full version''.
The number 1 in the code studyDYxxCScovYield.C argument list indicates
that the result file should be saved. The value of iMethod should be
1, 2, or 3. The code studyDYeeCScovYield.C also has a method 4. An
extra tag for the output file can be provided for this macro.

The macro studyDYxxCS creates the randomized ensemble and the
covariance matrix of the final uncertainties that could be used, if it
is good. The macro studyDYxxCScovYield creates the covariance matrix
from the ensemble by applying the specified method.
The macro studyDYxxCScovYieldMethods.C plots the
uncertainty values from the covariance matrices.

% ----------------------------------
\section{Calculation of the combined result}

The BLUE method is wrapped in the class \myund{BLUEResult_t}.

\textit{This is outdated since February 2017.}
The
combination is defined in the macro analyseBLUEResult.C that contains
only one function,\\
\indent
\myund{combineData(covEE_inp,covMM_inp,covEM_inp,outputFileTag,plotTag,
  $\backslash\backslash$\\
  \tab flagPrintCanvases)}\\
\noindent
 The function reads the central values
from embeded files, while the uncertainties are provided via input.
Its performance can be checked assuming no correlations (and taking
the uncertainties from the embeded in analyseBLUEResult.C input files)
by using the code \myund{work13TeV_noCorr.C}.

\textit{The latest code} used for the combination of the results is
\myund{work13TeV_corr.C}. It handles both the correlated case and
uncorrelated case. It also allows adjustment of the covariance matrix
to the externally-evaluated uncertainties.

A helper struct \myund{CovStruct_t}\ (defined in CovStruct.h) was
introduced to separated the uncertainties into four cathegories: the
statistical covariance of the observed yield (\myund{covStat_Yield}),
the statistical covariance not of the observed yield
(\myund{covStat_nonYield}), the systematic covariance of the acceptance
(\myund{covStat_Acc}), and the systematic covariance not of the
acceptance (\myund{covSyst_nonAcc}). The rationale for the division is
that the statistical uncertainty of the yield is a publishable
number. The statistical uncertainties can be evaluated by my codes,
thus the uncertainties can be evaluated properly. The systematic
uncertainty on the acceptance is correlated between the electron and
the muon channel. Finally, the systematic uncertainty on other
quantities is sometimes adjusted by hand.

% --------------------------------------------------------------
\section{Calculation of the efficiency scale factor systematic covariance}
\label{sect-effSFsyst}

\textit{Started July 27, 2017.}

I would like to be able to calculate the separate covariance matrices
of the event efficiency scale factor systematic uncertainties for each
source of the systematic uncertainty. Earlier, I had created a
cumbersome class \myund{DYTnPEff_t}\ that was able to keep each
source of the uncertainty separately. I would like to check that the
separate covariance matrices are calculable. If it is not implemented,
I should do this.

% -----------------------------------
\subsection{Studying previous codes}

\textit{The conclusion from this study} is that the macro
\myund{plotDYeeESF_coll.C}\ was used to produce the needed plots, and
the macros \myund{calcRhoRndVec_ee_syst.C}\ and
\myund{studyDYeeCS.C}\ were only partially developed towards the final
implementation. A full story is below.

I notice that I have already a
\textbf{macro \myund{calcRhoRndVec_ee_syst.C}}, so I look it up in
this document and see its description in section
\ref{sect-DielectronCov},
dated from Jan 19, 2017.

To check the macro, I ran \myund{calcRhoRndVec_ee_syst.C}\ with the
recommended flags (1,0,1,1111), standing for nToys=1, testCase=0,
uncorrFlag=1,\linebreak limitToSrc=1111. Compiling the code, I noticed that
there are several parameters that are not connected. They belong to
the commented out body of a function randomizeSF. Probably, I intended
to implement the covariance calculation, but either did not finish the
task, or decided to use just the diagonal uncertainties.

To be able to run the code, I needed input files with efficiency
systematics \myund{dyee_effSyst_TYPE}\ (where TYPE is RECO, ID, or
HLT). I took them from an earlier instance of the package in the
directory DYee2015-20161129-effSyst on ``supercomp'' SL6b-local. This
was the latest version and, most likely, the one that was used in my
reported evaluation of the systematic uncertainty. The flag ``1111''
just short-cuts to a restricted block of instructions.

The run also needs \myund{dyee_test_dressed_El3.root}. I noticed that
my latest instance of the DYee2015, DYee2015-20170527-combi-false used
the version named \myund{dyee_test_dressed_ElMay2017.root}\ and also
``May2017false''. Those files are produced by my own macro,
\myund{processDYee_dressed.C}, and here is no difference between those
two versions. The two files are produced only for the convenience of
identifying variations of other inputs, most notably, of the DYee
cross section file \myund{cs_DYee_13TeV_ElXX.root}.
The version ``May2017'' has RooUnfoldResponse
objects as Ridhi has set them up (with the flag
UseOverflow=true). While the version ``May2017false'' has this flag
set to false (this should be the correct behavior). I copy the needed
files, namely,
\myund{dyee_test_dressed_ElMay2017*.root}\
and\linebreak
\myund{cs_DYee_13TeV_ElMay2017*.root},
and change the variable inpVersion in the code from
\myund{_verEl3}\ to \myund{_verElMay2017false}.

The flag limitToSrc=1111 makes the macro
\myund{calcRhoRndVec_ee_syst.C}\ to plot the alternative event
efficiency scale factors produced with the alternative tables obtained
by applying $+1\sigma$ variation on both data and simulation. I tried
to understand the reason for it. My presentation dated from
20161207-Juodagalvis-effSyst.pdf shows distributions with
$(\pm\sigma_\mathrm{data},\pm\sigma_\mathrm{MC})$.  The macro
\myund{calcRhoRndVec_ee_syst.C}\ saves only one file, when run with
the flag limitToSrc=1000, namely, dyee-rho-syst.root. This file
contains only 1 instance of the calculated average event efficiency
scale factor. Since I found this file in the directory
DYee2015-20161129-effSyst, it must the other mentioned macro
studyDYeeCS.C that was use to produce the plots for the presentation.

The \textbf{macro \myund{studyDYeeCS.C+(_varRhoSyst,0,0)}}\
required\linebreak
\myund{cs_DYee_13TeV_ElMay2017false.root} (so I updated what I copied
earlier). The flag \myund{_varRhoSyst}\ calls a function
createSystFile. However, it produces one-case plot.

I checked the presentation source file and the previous instance
DYee2015-20161129-effSyst, and finally identified that the plots were
produced by a \textbf{macro \myund{plotDYeeESF_coll.C}}. The plots
contain only $\avgrho$ distributions and not the cross sections or the
covariance matrices. I was reminded that each source of the systematic
uncertainty either affects data or simulation, as a result, variation
only of the relevant sigma is important.

\textit{Conclusion:} other macros were used in search for the final
implementation and the algorithm, but not for the presentation. That
is why they are not complete. The plots were produced by
\myund{plotDYeeESF_coll.C}.

% -----------------------------------
\subsection{Updating the macros}

It is clear now what I would like to achieve. I would like that
\myund{calcRhoRndVec_ee_syst.C}\ produces a set of randomized
distributions for each source of the uncertainty. Although each kind
of efficiency has to be randomized separately, the systematic
deviation should be applied at a random scale.

I started to work on \myund{calcRhoRndVec_ee_syst2.C}\ to fix bugs and
making sure the ideas are implemented correctly. In particular, I
moved efficiency-loading lines to \myund{DYmm13TeV_eff} and created a
function createEffCollection.

The final version of the macro \myund{calcRhoRndVec_ee_syst2.C}\ has
several integer arguments (nToys, rndProfileKind, limitToSrc,
saveRndRhoVec, recreateCollection), where nToys specified the number
of random samples to produce. rndProfileKind=0 for a real calculation
(the values are related to
\myund{DYTnPEffColl_t::TRandomizationKind_t}\, with 0 corresponding to
\myund{_rnd_ampl}). limitToSrc can be -1 (all sources), 1111 (plot
only), or from 0 to 3, corresponding to built-in sources of systematic
uncertainty (bkgPdf, sigPdf, NLOvsLO, and tag). saveRndRhoVec
indicates whether the randomized values of $\rho$ should be saved to a
file. recreateCollection should be 0 to produce the needed ROOT file
from sources of systematic uncertainties. Curently it is adjusted for
the electron channel only.

I have also introduced a new kind of \myund{TVaried_t}, namely,
\myund{_varRhoSystFile}, updated \myund{CrossSection_t}, and adjusted
\myund{studyDYeeCS.C}\ to work with the provided distributions of
$\rho$. The new variable works in the same way as \myund{_varRhoFile}.


I have everything implemented, however, the comparison with Ridhi
result is not very good. My evaluation is greater than hers. I will
test the muon case before drawing the conclusions.


\textit{Continued on Aug 03, 2017.} I have implemented the muon case
and the agreement between my evaluation of the uncertainty and the
uncertainty from Kyeongpil Lee is very similar. Thus I have to recheck
the electron channel.

\textit{Update on Aug 11, 2017.} Due to the mismatch in the evaluated
systematic uncertainty between myself and Ridhi, I communicated with
Ridhi and Fabrice Couderc. Ridhi has sent me the files with RECO, ID
and HLT efficiencies, that she was using to evaluate the systematics,
and I found no difference from my values while our evaluations differ
a lot. The largest difference in the efficiencies was coming from
RECO, so I was trying to verify that the efficiency scale factors are
correct.  I wrote to Fabrice about the contents of the files with RECO
efficiencies and their scale factors in her directory
\url{http://fcouderc.web.cern.ch/fcouderc/EGamma/scaleFactors/moriond2016_76X/reco/}
I was not sure about the differences between my evaluation of
systematic uncertainties and the values from Fabrice. It turned out,
the plots with systematics were done using the $\eta$-averaged
efficiencies (both data and MC), which has led to disagreement. I
introduced methods that do averaging of the 2D histograms.

The latest version of \myund{calcRhoRndVec_ee_syst2} has an additional
flag, whether the efficiencies should be symmetrized. It also tags
the files with this version. The macro should be run with values,
e.g.\ (2000,0,-1,1, 1 [recreate], 1 [symmetrize]).


% -----------------------------------------------------
\subsection{Restarting evaluation}
\textit{Started on Nov 28, 2017}

Working in DYee2015-20171128-rhoSyst-newCombi/Covariance on 'supercomp'.

\medskip

Since a new set of the uncertainties was released by Kyeongpil and
Ridhi, and I have to answer Ridhi about the ESF uncertainties, I
started from ESF systematics.

The code
\begin{verbatim}
.x calcRhoRndVec_ee_syst2.C+(200,0,1111,1,1,0)
\end{verbatim}
requires an input file \myund{dyee_effSyst_RECO.root}, and probably
the rest as well. This file is produced by a macro
\myund{plotDYeeESF.C}:
\begin{verbatim}
.x plotDYeeESF.C+("RECO",1,1,0) # showPlots=1,save=1,altSet=0(not RC)
\end{verbatim}
altSet=1 required \myund{RC_eleRECO_Final.txt}, so I decided to go
with my efficiencies, since as far as I recall, the values were the
same.


Having prepared \myund{dyee_effSyst_RECO.root},
\myund{dyee_effSyst_ID.root}, \myund{dyee_effSyst_HLT.root}, I ran the
macro \myund{calcRhoRndVec_ee_syst2.C}\ again (with $-1$ instead of
1111), and had to also copy the file
\myund{dyee_test_dressed_ElMay2017false.root}. Then I was able to
obtain the randomized rho's, but because of the flag, different
sources of systematic uncertainties were not separated.

The macro created a file dyee-rho-syst2.root, containing several
histograms:
\begin{verbatim}
TFile**		dyee-rho-syst2.root
 TFile*		dyee-rho-syst2.root
  KEY: TH1D	h1rho_stat;1	h1rho_stat
  KEY: TH1D	h1rho_bkgPdf;1	h1rho_bkgPdf
  KEY: TH1D	h1rho_sigPdf;1	h1rho_sigPdf
  KEY: TH1D	h1rho_NLOvsLO;1	h1rho_NLOvsLO
  KEY: TH1D	h1rho_tag;1	h1rho_tag
  KEY: TObjString	timeTag;1	Collectable string class
\end{verbatim}

This file is subsequently used by studyDYeeCS.C. To be able to run as
\myund{studyDYeeCS.C+(_varRhoSyst,10,0)} (nSamples=10, save=0), I had
to copy the file \myund{cs_DYee_13TeV_ElMay2017false.root}.

Since my uncertainties are smaller than those provided by Ridhi, I
would like to make sure that I correctly include the sources. A
possible problem would be that I include only one kind of systematic
uncertainty for a particular source (e.g.\ only RECO uncertainty for
tag dependency).

The function is createEffCollection, defined in \myund{DYmm13TeV_eff.cc}.
I could not find any problem with effCollection.
It seems that all alternative tables are included as they are supposed
to.

Another question is whether
\myund{coll.randomizeByKind(int(DYTnPEffColl_t::_rnd_ampl),tag,iSrc,1,1);}\
indeed produces an alternative table. I looked at the numbers, they
seem reasonable (the checked cases agree with the input txt file).

\textit{Continued from Nov 29, 2017.}

I decided to check my contributions. I wrote a new methods with the
top-most \myund{EventSpace_t::listContributionsToAvgSF_v2}\ that
relies on \myund{EventSpace_t::calculateScaleFactor_contr}.
Eventually I got suspicious about my code. Resolving indices and
ranges was successful. I use \myund{EventSpace_t::fill}\ with
unmodified flat indices which would return 0 for the lowest
$(p_T,\eta)=(-2.5..-2.0,10-20)$ bin. The maximum unmodified flat index
value 49 is for the highest bin $(2.0-2.5,50-2000)$. Since the axes of
\myund{EventSpace_t::fh2ESV}\ are from $-0.5$ to
$\mathrm{DYtools::EtaPtFIMax}+0.5$, which is from $-0.5$ to $51.5$, I
should disregard overflowing values.

The conclusion was that I cannot see a problem in my code.



% ------------------


\end{document}

