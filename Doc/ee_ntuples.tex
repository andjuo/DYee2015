\documentclass[11pt]{article}

\usepackage{hyperref}
\usepackage{xstring} % StrSubstitute


%\newcommand{\cosh}{\mathrm{cosh}}

\def\myund#1{%
  \saveexpandmode\expandarg
  \IfSubStr{#1}{_}{%
   \StrSubstitute{#1}{_}{\_}}{#1}%
  \restoreexpandmode
}


%opening
\title{Electron energy calibration in CMSSW 76X}
\author{A.~Juodagalvis}
\date{started May 17, 2017}

\begin{document}


\maketitle

\begin{abstract}
Description of electron n-tuples produced by Arun.
\end{abstract}

\section{Introduction}

The package DYee2015 already had a branch DYee-cov-input-skim with
macros to simplify Arun's n-tuples with the signal samples. The macro
treeRead1.C placed in SkimEE directory is using the structure defined
in \myund{dyee_tree_v2.h}. (A more detailed check revealed that this
structure is probably related to the ``compressed'' n-tuple that I
asked Ridhi to provide me last year, since the signal n-tuples contain
a slightly different structure.}

I took data samples from\\
\myund{/store/group/phys_higgs/cmshww/arun/DYAnalysis_76X_Calibrated/Data}. The
files are named \myund{SE_2015.root}, \myund{Photon_2015.root}, and
\myund{MuEG_2015.root}. Using supposedly SingleElectron data file
\myund{SE_2015.root}, I created a class \myund{dyee_data_t.h}.

Having created a script createTreeClass.C, I have verified that the
same tree structure is available in simulated samples.

Modifications to the created class. (1) Preparation actions. I renamed
\myund{dyee_data_t.C}\ to \myund{dyee_data_t.cc}. The header file
contained unneeded includes of $<$vector$>$ and ``vector.dll''. The
method Cut contained unused parameter 'entry', so I added one 'if'
statement that should be optimized out by the compiler, but would
create no warning. (2)



\end{document}
