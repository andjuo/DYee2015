\documentclass[11pt]{article}

\usepackage{hyperref}
\usepackage{xstring} % StrSubstitute


%\newcommand{\cosh}{\mathrm{cosh}}

\def\myund#1{%
  \saveexpandmode\expandarg
  \IfSubStr{#1}{_}{%
   \StrSubstitute{#1}{_}{\_}}{#1}%
  \restoreexpandmode
}


%opening
\title{Electron energy calibration in CMSSW 76X}
\author{A.~Juodagalvis}
\date{started July 27, 2016}

\begin{document}


\maketitle

\begin{abstract}
	Electron energy corrections are described in the twiki \url{https://twiki.cern.ch/twiki/bin/viewauth/CMS/EGMSmearer}. Taking the case of electron calibration for CMSSW 76X, I try to understand how the uncertainty should be propagated.
\end{abstract}

\section{Introduction}

The instructions for electron energy calibration in CMSSW 76X are pretty simple:
\begin{verbatim}
content...cmsrel CMSSW_7_6_3_patch2
cd CMSSW_7_6_3_patch2/src
cmsenv
git cms-merge-topic -u matteosan1:smearer_76X
scram b
\end{verbatim}

The configuration file should provide information for the producer:
\begin{verbatim}
process.load('EgammaAnalysis.ElectronTools.calibratedElectronsRun2_cfi')
.......
correctionType = "76XReReco"
.......
calibratedPatElectrons = cms.EDProducer("CalibratedPatElectronProducerRun2",

# input collections
electrons = cms.InputTag('slimmedElectrons'),
gbrForestName = cms.string("gedelectron_p4combination_25ns"),

# data or MC corrections
# if isMC is false, data corrections are applied
isMC = cms.bool(False),

# set to True to get special "fake" smearing for synchronization. Use JUST in case of synchronization
isSynchronization = cms.bool(False),

correctionFile = cms.string(files[correctionType])
)
\end{verbatim}

The coefficients file (files[correctionType]) are defined in the cfi
file python/calibratedElectronsRun2\_cfi.py. The file basename is
data/76X\_16DecRereco\_2015. There are four files in this directory,
starting with this basename: Etunc\_scales.dat, Etunc\_smearings.dat,
\_scales.dat, and \_smearings.dat. The 'scales' files are probably for
data, and 'smearings' are for MC.

The producer is defined in the file
plugins/CalibratedElectronProducersRun2.cc. It is quite simple. It
takes electrons from the input collection, calls the energy corrector
of type ElectronEnergyCalibratorRun2, defined in
\{interface/src\}ElectronEnergyCalibratorRun2, and saves the electron
with the corrected energy.

The ElectronEnergyCalibratorRun2 has four data members:
epCombinationTool, isMC(bool), synchronization(bool) -- should be
false in the real analysis, and rng(TRandom).

The calibrator is called in ln68 of
CalibratedElectronProducersRun2.cc:
calibrate(pat::Electron\&,runNumber,iEvent.streamID())). In some cases,
the electron object class may be reco::GsfElectron. The header file
for ElectronEnergyCalibratorRun2 states, that "StreamID is needed when
used with CMSSW Random Number Generator." The default value is
StreamID::invalidStreamID.

The method
ElectronEnergyCalibratorRun2::calibrate(reco::GsfElectron\&)
performs simple actions.
The electron, whether it is reco::GsfElectron or pat::Electron
(descendant of reco::GsfElectron), is copied to an instance of
SimpleElectron, which is calibrated and written back to the original
object (ln36-38 of ElectronEnergyCalibratorRun2.cc).

The calibration of the SimpleElectron object is performed in the
method ElectronEnergyCalibratorRun2::calibrate(SimpleElectron\&). The
calibration is handled by the \_correctionRetriever member that has a
type of EnergyScaleCorrection\_class. The retriever is created by
giving the name of the file with corrections (actually, the base name,
since the file name is
'correctionFileName'+'\_\{scales$\vert$smearings\}.dat').

The auxiliary class correctionValue\_class is defined in
interface/EnergyScaleCorrection\_class.h. This class is used to read
energy scale and smearings from .dat files.The implementation is in
src/EnergyScaleCorrections\_class.cc. The constructor of the
EnergyScaleCorrection\_class takes two arguments, namely: the input
file name base (the actual file is +'\_scales.dat' and
+'\_smearings.dat'), and genSeed. The default values for correction
indicators doScale and doSmearings are 'false', and smearingType is
set to ECALELF. The constructor reads data from the file by calling
ReadFromFile and ReadSmearingFromFile. The scale file format is
'category' (like \myund{absEta_0_1-highR9-Et_0_20}),
String='runnumber', runMin, runMax, deltaP, \myund{err_deltaP},
\myund{err_deltaP_stat}, \myund{err_deltaP_syst}\ (according to the
code Ln.166-177). The field \myund{err_deltaP}\ is ignored, i.e.\ it
is not used when calling AddScale(category, runMin, runMax, deltaP,
\myund{err_deltaP_stat}, \myund{err_deltaP_syst}). The implementation
of the function
EnergyScaleCorrection\_class:ReadSmearingFromFile(TString) starts in
Ln264 of \myund{EnergyScaleCorrection_class.cc}. The smearingType
value is set to ECALELF in the constructor. Thus the code reads-in
category, Emean, \myund{err_Emean}, rho, \myund{err_rho},
\myund{phi_string}, and \myund{err_phi_string}. The values of phi are
read as strings. The file
\myund{76X_16DecRereco_2015_smearings.dat}\ has values \myund{M_PI_2}\
which mean $\pi/2$, according to $<$math.h$>$. Smearings are saved by
calling \myund{AddSmearing(category, runMin, runMax, rho, err_rho, phi,
err_phi, Emean, err_Emean)}.

The type of activated calibration is set in the constructor of
ElectronEnergyCalibratorRun2. If the argument isMC is true, the
correction retriever members (doScale,doSmearings) are set as
(false,true). Otherwise, the values are set to (true,false).
The calibration method calibrate(SimpleElectron\&) first checks, that
the electron flag ``from MC'' matches its own flag isMC.

The calibration is performed by taking the absolute value of
SimpleElectron::getEta(), aeta=$|\eta|$, and recalculating the
corresponding $E_T$ value as
simpleElectron::getNewEnergy()/$\cosh(|\eta|)$.  Since the function
$\cosh(x)=\frac12(e^{x}+e^{-x})$ is symmetric, the initial sign of
$\eta$ has no effect.  The header file indicates that the function
getNewEnergy() simply returns the value of the field newEnergy.  What
all that implies for our gsf::Electron? The SimpleElectron object is
constructed by calling the constructor (electron,runNumber,isMC). This
constructor is available in the code if the preprocessor directive
SimpleElectron\_STANDALONE is active and has the implementation lines
in SimpleElectron.cc file.  The value of newEnergy is copied from
regEnergy, while regEnergy is the value of
reco::GsfElectron::correctedEcalEnergy(). In addition, the value of
$\eta$ is taken from the field GsfElectron::superCluster()-$>$eta().

The object correctionRetriever is called to get scale or smearing
correction (ScaleCorrection or getSmearingSigma) with the arguments of
(getRunNumber(),isEB(),getR9(),aeta,et[,0,0]). During the construction of
SimpleElectron, the value of isEB is taken from GsfElectron::isEB(),
and the value of r9 is taken from GsfElectron::full5x5\_r9().

EnergyScaleCorrection\_class::ScaleCorrection(runNumber,isEBEle,R9Ele,etaSCEle,EtEle)
(Ln158-159 in EnergyScaleCorrection\_class.h, Ln51 in .cc file) is the
method to get energy scale correction to measured electrons. If
doScale=false, it returns 1, otherwise it calls the method
getScaleOffset(runNumber,...) (the arguments are exactly the same) and
returns its value. Note that here is also a \textit{public} method
ScaleCorrectionUncertainty (with the same arguments) that returns the
total uncertainty for a given $(|eta| \times \mathrm{R9}$ category.
The function getScaleOffset first creates the category of a type
\myund{correctionCategory_class} from the provided arguments
(runNumber, etaSCEle, R9Ele, EtEle). Then it tries to locate the
category in the variable scales which is of type
map$<$\myund{correctionCategory_class, correctionValue_class}$>$
(defined by typedef as \myund{correction_map_t}). If the category is
found, the method returns the value 'scale' of the second element
(iterator-$>$second.scale). The tricky part is how the non-discrete
values are handled when checking the category. The
\myund{correctionCategory_class}\ defines ranges of validity for
runNumber, R9, Et, eta. The constructor(TString category) converts the
provided string to the ranges. The category of the individual electron
is created by calling a constructor
\myund{correctionCategory_class}(runNumber,etaEle, R9Ele,EtEle). The
method \myund{correctionCategory_class}::operator$<$ compares the
values of the individual fields (runNumber, eta, r9, et) in order to
determine whether the value of this operation is 'true' or
'false'. There might be some tricky cases if both objects would have
defined ranges as compared to a 'simple' case, when one object has
different values of the range edges and the other object has value
intervals of a zero length.

ElectronScaleCorrection\_class::getSmearingSigma(runNumber,isEBEle,R9Ele,etaSCEle,EtEle,nSigma\_rho,nSigma\_phi)
(Ln179 of the .h file, Ln353 of the .cc file). (A reminder: the method
is called by the corrector with nSigma\_rho=nSigma\_phi=0.)  The
functions find the smearing definition by the category created from
(runNumber, etaSCEle, R9Ele, EtEle). Rho correction is defined as
\myund{rho + rho_err$\times$ nSigma_rho}, and phi correction is
defined as \myund{phi + phi_err$\times$ nSigma_phi}. Since the method
is called with \myund{nSigma_rho=nSigma_phi=0}, the corrections are
actually rho and phi. The constant term is $\rho \sin(\phi)\equiv
\rho$, since our $\phi=\pi/2$, and alpha is
$\rho\timesE_\mathrm{mean}\times\cos{\phi}\equiv0$. Note that $\alpha$
is 0 because of both Emean=0 and $\phi=\pi/2$. Thus the used smearing
values are defined in categories of $|eta|$ and low/high-R9.  The
smearing sigma is defined as
$\sqrt{\mathrm{constTerm}^2+\alpha^2/\mathrm{EtEle}}\equiv
\mathrm{constTerm}\equiv\rho$.  This value would also be obtained by
calling the method
\myund{EnergyScaleCorrection_class::getSmearingRho}(runNumber,
isEBEle, R9Ele, etaSCEle, EtEle). The method getSmearingSigma also
allows to vary the smearing sigma by the uncertainty of the factor
$\rho$ (\myund{nSigma_rho} should be set e.g.\ to 1).

Now returning to the method
ElectronEnergyCalibratorRun2::calibrate(SimpleElectron\&). We learned
how the values of scale and smear are obtained from the
correctionRetriever object. If the electron is simulated and has to be
smeared, the correction factor is calculated as $\mathrm{corr}\equiv(1
+ \mathrm{smear} \times \mathrm{gauss}(\mathrm{streamID}))$. The new
energy (SimpleElectron::getNewEnergy()) is multiplied by this factor
(but not assigned yet!), and its uncertainty
$\sqrt{(\mathrm{getNewEnergyError} \times \mathrm{corr})^2 +
  (\mathrm{smear}\times \mathrm{getNewEnergy}\times\mathrm{corr})^2}$
are calculated by calling std::hypot(x,y) that ``computes the square
root of the sum of the squares of x and y, without undue overflow or
underflow at intermediate stages of the computation'' (according to
cppreference). [This uncertainty calculation is not clear to
  me, since the uncertainty propagation formula for the expression
  $e'=(1+\sigma)e$ would give $(\delta e')^2=(1+\sigma)^2(\delta
  e)^2+e^2\sigma^2$. (Here I employed $\delta(1+\sigma)=\sigma$.)
  Whereas the code uses the expression $(\delta
  e')^2=(1+\sigma)^2(\delta e)^2 + (e')^2\sigma^2$.]
The scaling is performed in a similar way, the new value of ECAL
energy is calculated from
$\mathrm{SimpleElectron::getNewEnergy()}\times\mathrm{scale}$, and the
uncertainty is evaluated from
hypot(getNewEnergyError()\,$\times$\,scale, smear $\times$
modified\_energy). [This uncertainty is not clear to me. First of all,
is it obvious that the 'smear' parameter would return the uncertainty
of the 'scale' factor.]

Subsequently, the SimpleElectron object electron is updated by calling
setNewEnergy and setNewEnergyError (both are simple-assignment
methods) and then invoking epCombinationTool to combine the electron
fields. EpCombinationTool is defined in
ElectronTools/interface/EpCombinationTool.h. The object is initialized
in the ElectronEnergyCalibratorRun2 constructor by using the pointer
value provided to the constructor. In our code, this pointer is
obtained from edm::EventSetup instance iSetup, by getting the product
named theGBRForestName. The configuration file
\myund{python/calibratedElectronsRun2_cfi.py}\ defined gbrForestName
as '\myund{gedelectron_p4combination_25ns}'.

The method EpCombinationTool::combine(SimpleElectron\&) takes the both
new energy and its error of the electron (getNewEnergy() and
getNewEnergyError()) and recomputes several quantities:
regressionInputs[1]= energyRelError= energyError/energy,
regressionInputs[3]= momentumRelError= momentumError/momentum,
regressionInputs[4]= errorRatio= energyRelError/momentumRelError,
regressionInputs[5]= eOverP= energy/momentum, regressionInputs[6]=
eOverPerror (calculated correctly). The other needed values are
energy, momentum=electron.getTrackerMomentum(),
momentumError=electron.getTrackerMomentumError(), and the additional
needed flags are isEcalDriven, isTrackerDriven, electronClass, and
isEB. Based on regressionInput values, the weight is obtained from
\myund{m_forest}-$>$GetResponse(regressionInputs). The combined
momentum is calculated from the tracker momentum and the energy as
$\mathrm{weight} \times p_\mathrm{tracker} + (1-\mathrm{weight})\times
E_\mathrm{ECAL}$. The pure tracker electrons have track momentum error
of 999, the correction is not applied to them. Other electrons get the
values of the combinedMomentum and the combinedMomentumError
(calculated correctly).

The electron with the combined momentum is converted back to
reco::GsfElectron by method SimpleElectron::writeTo (see Ln38 of
ElectronEnergyCalibratorRun2.cc). The method writeTo is implemented in
Ln27 of SimpleElectron.cc.

\end{document}
